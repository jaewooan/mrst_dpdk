\documentclass[11pt]{amsart}
\usepackage{amsmath, amsfonts, amssymb, amsthm, graphicx}

\usepackage[backend=biber,style=authoryear]{biblatex}
\addbibresource{sprefs.bib}

\usepackage{xspace}
\usepackage{tikz}
\usepackage{color}
\usepackage{amsmath}
\usepackage{amsfonts}
\usepackage{latexsym}
\usepackage{graphicx}
\usepackage{bm}

% 
% =======================================================================
% Formatting tools
% =======================================================================
% Gentle spacing after macros
% Reference: The LaTeX Companion, p.50
\newcommand{\Real}{\mathbb R}
\newcommand{\eg}{e.g.,\xspace}
\newcommand{\ie}{i.e.,\xspace}
\newcommand{\etc}{etc.\@\xspace}
\newcommand{\etal}{et~al.\@\xspace}
% 
% 
% =======================================================================
% Common Mathematics
% =======================================================================
\newcommand{\vect}[1]{\boldsymbol{#1}}
\newcommand{\mat}[1]{\boldsymbol{#1}}
\newcommand{\transp}[1]{{#1}^{\ensuremath{\mathsf{T}}}}
% 
% =======================================================================
% Math Operators
% =======================================================================
\DeclareMathOperator{\erfc}{erfc}
\DeclareMathOperator{\spanop}{span}
\DeclareMathOperator{\rank}{rank}
\DeclareMathOperator{\tr}{tr}
\DeclareMathOperator{\diag}{diag}
\DeclareMathOperator{\Grad}{Grad}
\DeclareMathOperator{\Dive}{Div}
% 
\DeclareMathOperator{\GRAD}{\scshape Grad}
\DeclareMathOperator{\DIVE}{\scshape Div}
% 
\newcommand{\grad}{\nabla}
\newcommand{\dive}{\nabla\cdot}
\newcommand{\curl}{\nabla\times}


\newcommand{\cads}{c^a}
\newcommand{\cpads}{c_p^a}
\newcommand{\csads}{c_s^a}
\newcommand{\chatads}{\hat{c}^a}
\newcommand{\eff}{\mathrm{eff}}
\newcommand{\cpmax}{c_{p,\max}}
\newcommand{\Sb}{{\bm{S}}}
\newcommand{\cb}{{\bm{c}}}
\newcommand{\Kb}{{\bm{K}}}
\newcommand{\kb}{{\bm{k}}}
\newcommand{\nb}{{\bm{n}}}
\newcommand{\ub}{{\bm{u}}}
\newcommand{\ubmax}{\ub_{\max}}
\newcommand{\abs}[1]{\left| #1\right|}
\newcommand{\norm}[1]{\left\| #1\right\|}

\newcommand{\fracpar}[2]{\frac{\partial #1}{\partial #2}}
\newcommand{\pref}{p_\text{ref}}
\newcommand{\surf}{\text{surf}}
\newcommand{\nosurf}{\text{nosurf}}

\usepackage{soul}
% Fix found for highlighting support (when adding comments)
\soulregister\cite7
\soulregister\ref7
\soulregister\eqref7
\soulregister\num7
\soulregister\pageref7

\setul{0mm}{1mm}
\setstcolor{red}
\sethlcolor{yellow}

\def\mytilde{\hspace*{-2pt}
  \begin{tikzpicture}
    \draw[red] (0, 0) .. controls (1pt, 1pt) .. (2pt, 0) .. controls +(1pt, -1pt) .. +(2pt, 0);
  \end{tikzpicture}
}
\newbox\mybox
\setbox\mybox\hbox{\mytilde}
\makeatletter
\def\xavierul #1{\begingroup\def\SOUL@ulleaders{\leaders\hbox{\raisebox{-4pt}[0pt][0pt]{\hbox to 5pt {\copy\mybox}}}\relax}\ul{#1}\endgroup}
\makeatother
\colorlet{softblue}{blue!20!yellow!30!white}

\def\xaviercom #1{\begingroup\protect\sethlcolor{softblue}\hl{#1}\endgroup}
\def\xavieradd #1{\begingroup\protect\sethlcolor{softblue}\textcolor{red!40!black}{\hl{#1}}\endgroup}
\def\xavierst #1{\st{#1}}

\def\xavierrep #1#2{\st{#1} \xavieradd{#2}}
\def\xavierulcom #1#2{\xavierul{#1} \xaviercom{#2}}

\setlength{\parindent}{0cm}
\setlength{\parskip}{5mm}

\begin{document}

\title[Surfactant Polymer model]{Surfactant Polymer model}
\author{Xavier Raynaud and Xin Sun}

\maketitle


We consider a general system with $n_c$ components and $n_p$ fluid phases and one solid phase, which is used to account for the adsorption processes. We denote by $x_{i,\alpha}$ the mass fraction of component $i$ in the phase $\alpha$. We denote by $v_\alpha$, $\rho_\alpha$, and $s_\alpha$ the velocity, density and saturation of the phase $\alpha$. The mass conservation for the component $i$ is given by
\begin{equation}
  \label{eq:massconsgen}
  \fracpar{}{t}(\phi \sum_{\alpha = 1}^{n_p}(\rho_\alpha x_{i,\alpha}s_\alpha) + \rho_s(1 - \phi)x_{i,s}) + \dive(\sum_{\alpha = 1}^{n_p}\rho_\alpha x_{i,\alpha}v_\alpha) = q_i
\end{equation}
In the case of the surfactant-polymer model, we have five components, $i\in\{O, G, W, P, S\}$, and
four phases, $\alpha\in\{o, g, ww, wp\}$. We assume $\rho_{ww}=\rho_{wp}=\rho_{w}$. We denote
\begin{equation*}
  s_w = s_{ww} + s_{wp}.
\end{equation*}
To model the viscosity change of the mixture, we firstly consider the impact of
surfactant. The change in viscosity by surfactant is only modeled for the water
phase by introducing a multiplier, which is a function of the surfactant
concentration. The viscosibility, that is, the dependence of the viscosity with
respect to pressure, is introduced as a multiplier $\gamma(p)$,
\begin{equation}
  \label{eq:defmuw}
  \frac{\mu_w(p)}{\mu_w^0} = \gamma(p),
\end{equation}
The effect of surfactant on viscosity is also introduced as a multiplier
function, $\gamma_s(c_s)$.  At reference pressure condition, we have
\begin{equation}
  \label{eq:defmuw}
  \frac{\mu_w(c_s)}{\mu_w^0} = \gamma_{s}(c_s),
\end{equation}
After that, we use the Todd--Longstaff model \cite{TL72:jpt} to model the
viscosity change by polymer. This model introduces a mixing parameter
$\omega\in[0, 1]$ that takes into account the degree of mixing of polymer into
water. The polymer effect on the viscosity is included in the model as a
multiplier. For a solution without surfactant and at reference pressure
condition, we have
\begin{equation*}
  \frac{\mu_w(c_p)}{\mu_w^0} = \gamma_p(c_p)
\end{equation*}
for some given multiplier function $\gamma$. Still in absence of the surfactant
and at reference pressure condition, the viscosity of the phase $ww$ and $wp$ are given by
\begin{equation*}
  \frac{\mu_{ww}(c_p)}{\mu_w^0} = \gamma_p(c_p)^\omega\gamma_p(\cpmax)^{1-\omega}
\end{equation*}
and
\begin{equation*}
  \frac{\mu_{wp}(c_p)}{\mu_w^0} = \gamma_p(c_p)^\omega\gamma_p(0)^{1-\omega},
\end{equation*}
or
\begin{equation*}
  \frac{\mu_{wp}(c_p)}{\mu_w^0} = \gamma_p(c_p)^\omega,
\end{equation*}
because $\gamma_p(0)=1$. When we consider the system including all effects
(surfactant, polymer and reservoir pressure conditions), these expressions
become
\begin{equation}
  \label{eq:defmuwp}
  \frac{\mu_{wp}(p, c_p, c_s)}{\mu_w^0} = \gamma(p)\cdot\gamma_s(c_s)\cdot\gamma_p(c_p)^\omega\cdot\gamma_p^{1-\omega}(\cpmax).
\end{equation}
and
\begin{equation}
  \label{eq:defmuww}
  \frac{\mu_{ww}(p, c_p, c_s)}{\mu_w^0} = \gamma(p)\cdot\gamma_s(c_s)\cdot\gamma_p(c_p)^\omega.
\end{equation}
For the Todd-Longstaff model, we assume that the volume of the two water-polymer
phases split with the ratio given by $\frac{c_p}{\cpmax}$, that is
\begin{equation*}
  s_{ww} = \left(1 - \frac{c_p}{\cpmax}\right)s_w \quad\text{ and }\quad s_{wp} = \frac{c_p}{\cpmax}s_w 
\end{equation*}
Then, the model uses the following expressions for the mass ratios and the velocities,
\begin{align*}
  x_{O,o} &= s_o,    & x_{O,g} &= s_g r_v & x_{O,ww} &= 0,      & x_{O,wp} &= 0,     & x_{O,s} &= 0,  \\
  x_{G,o} &= s_or_s, & x_{G,g} &= s_g,    & x_{G,ww} &= 0,      & x_{G,wp} &= 0,     & x_{G,s} &= 0,  \\
  x_{W,o} &= 0,      & x_{W,g} &= 0,      & x_{W,ww} &= s_{ww}, & x_{W,wp} &= s_{wp},& x_{W,s} &= 0, \\
  x_{P,o} &= 0,      & x_{P,g} &= 0,      & x_{P,ww} &= 0,      & x_{P,wp} &= c_ps_{w},& x_{P,s} &= c_p^a(c_p), \\
  x_{S,o} &= 0,      & x_{S,g} &= 0,      & x_{S,ww} &= c_ss_{ww}, & x_{S,wp} &= c_ss_{wp},& x_{S,s} &= c_s^a(c_s),
\end{align*}
and
\begin{align*}
  &v_o = -\frac{k_{ro}(s_o,c_s)}{s_o\mu_o}K(\grad p - \rho_o g \grad z),\\ 
  &v_g = -\frac{k_{rg}(s_g)}{s_g\mu_g}K(\grad p - \rho_g g \grad z),\\
  &v_{ww} = -\frac{k_{rw}(s_w,c_s)}{s_w\mu_{ww}(p, c_s, c_p)R_k(\cpads)}K(\grad p - \rho_w g \grad z),\\
  &v_{wp} = -\frac{k_{rw}(s_w,c_s)}{s_w\mu_{wp}(p, c_s, c_p)R_k(\cpads)}K(\grad p - \rho_w g \grad z).
\end{align*}
The function $R_k(\cpads)$ denotes the actual resistance factor and is a
non-decreasing function which models the reduction of the permeability of the
rock to the water phase due to the presence of absorbed polymer. The
concentration of absorbed polymer is denoted by $\cpads$. From these modeling
assumptions and \eqref{eq:massconsgen}, we obtain the following governing
equations,
\begin{subequations}
\begin{multline}
  \fracpar{}{t}(\phi(\rho_o s_o + \rho_g s_g r_v)) - \\ \dive\left(\rho_o\frac{k_{ro}(s_o,c_s)}{\mu_o}K(\grad p - \rho_o g \grad z) + \rho_gr_v\frac{k_{rg}(s_g)}{\mu_g}K(\grad p - \rho_g g \grad z)\right) = q_o,
\end{multline}
\begin{multline}
  \fracpar{}{t}(\phi(\rho_g s_g + \rho_o s_o r_s)) - \\ \dive\left(\rho_g\frac{k_{rg}(s_g)}{\mu_g}K(\grad p - \rho_g g \grad z) + \rho_or_s\frac{k_{ro}(s_o,c_s)}{\mu_o}K(\grad p - \rho_o g \grad z)\right) = q_g,
\end{multline}
\begin{multline}
  \fracpar{}{t}(\phi\rho_w s_w ) -
  \dive\left(\rho_w\frac{k_{rw}(s_w,c_s)}{\mu_{w,\eff}(p, c_s, c_p)R_k(\cpads)}K(\grad p -
    \rho_w g \grad z)\right) = q_w,
\end{multline}
\begin{multline}
  \fracpar{}{t}\left(\phi\rho_w s_w c_p + (1 - \phi)\rho_s c_p^a(c_p)\right) - \\ 
  \dive\left(\rho_w c_p\frac{k_{rw}(s_w,c_s)}{\mu_{wp}(p, c_s, c_p)R_k(\cpads)}K(\grad p -
    \rho_w g \grad z)\right) = q_p,
\end{multline}
\begin{multline}
  \fracpar{}{t}\left(\phi\rho_w s_w c_s + (1 - \phi)\rho_s c_s^a(c_s)\right) - \\ \dive\left(\rho_w c_s\frac{k_{rw}(s_w,c_s)}{\mu_{w,\eff}(p, c_s, c_p)R_k(\cpads)}K(\grad p - \rho_w g \grad z)\right) = q_s
\end{multline}
\end{subequations}
where $\mu_{w,\eff}$ is defined by
\begin{equation}
  \label{eq:defmuweff}
  \frac{1}{\mu_{w,\eff}}=\frac{1-c_p/\cpmax}{\mu_{ww}}+\frac{c_p/\cpmax}{\mu_{wp}}.
\end{equation}
Here, the relative permeabilities $k_{r\alpha}$ depends on the saturations and the surfactant concentrations except for the gas phase which is only related to the saturation. The phase pressures are given by the relation
\begin{equation}
  \label{eq:pressurerelationship}
  p_o = p_w + p_c(s_w, c_s) 
\end{equation}
where the capillary pressure function $p_c$ depends on the water saturation and surfactant concentration. The model for relative permeability follow Eclipse's implementation, see \cite{eclipse, Jorgensen}. First, we introduce the interfacial tension $\sigma$. At the pore level, the effect of surfactant is to modify the interfacial tension so that $\sigma$ is a function of the surfactant concentration, $\sigma(c_s)$. At the upscaled level, the change in relative permeability depends on the capillary number $N_c$ which measures the ratio between the viscous and capillary forces and is defined as
\begin{equation}
  \label{eq:defNc}
  N_c = \frac{\abs{K\grad p}}{\sigma}.
\end{equation}
Thus, the relative permeability is given as $k_{r\alpha}(s_\alpha, N_c)$. The values of $k_{r\alpha}(s_\alpha, N_c)$ are computed by interpolating the relative permeability curves for minimal and maximal $N_c$, in the following way. Let $N_c^{\nosurf}$ and $N_c^{\surf}$ denote the minimal and maximal values of the capillary numbers for which the relative permeability curves are given as
\begin{equation*}
  k_{r\alpha}^{\nosurf}(s_\alpha)\quad\text{and}\quad k_{r\alpha}^{\surf}(s_\alpha).
\end{equation*}
We denote the residual saturation values, for each phase, as $s_{\alpha. res}^{\nosurf}$ and $s_{\alpha. res}^{\surf}$. We use a logarithmic interpolation and we define the interpolation factor
$m$ as
\begin{equation} 
  \label{eq:defm}
  m(N_c) = \frac{\log_{10}(N_c) - \log_{10}(N_c^{\nosurf})}{\log_{10}(N_c^{\surf}) - \log_{10}(N_c^{\nosurf})},
\end{equation}
which belongs to $[0,1]$. The residual saturation values for the relative permeabilities $k_{r\alpha}(s_\alpha, N_c)$ is obtained by interpolation
\begin{equation}
  \label{eq:srhat}
  s_{\alpha,res}^{N_c} = m(N_c)\,s_{\alpha,res}^{\surf} + (1 - m(N_c))\,s_{\alpha,res}^{\nosurf}.
\end{equation}
The relative permeabilities $k_{r\alpha}^{\surf}$ and $k_{r\alpha}^{\nosurf}$
are rescaled to match these endpoints. We detail the construction only for
$k_{r\alpha}^{\surf}$ as the construction for $k_{r\alpha}^{\nosurf}$ is
completely similar. We define the function $k_{r\alpha}^{\surf}$ as
\begin{equation}
\label{eq:rescaledkr}
k_{r\alpha}^{\surf}(s_\alpha, N_c) = k_{r\alpha}^{\surf}(s_\alpha^{\surf}(N_c)),
\end{equation}
where $k_{r\alpha}^{\surf}$ is a given function (see above) and $s_\alpha^{\surf}(N_c)$ is defined through the relation
\begin{equation}
  \label{eq:defsnc}
  \frac{s_\alpha^{\surf}(N_c) - s_{\alpha,res}^{\surf}}{1 - s_{\alpha,res}^{\surf} - s_{\alpha',res}^{\surf}} =   \frac{s_\alpha - s_{\alpha,res}^{N_c}}{1 - s_{\alpha,res}^{N_c} - s_{\alpha',res}^{N_c}}.
\end{equation}
In \eqref{eq:defsnc}, $\alpha'$ denotes the phase which is not the phase $\alpha$. Similarly we define $k_{r\alpha}^{\nosurf}$. Finally, we define the relative permeability $k_{r\alpha}$ by interpolation using the coefficient $m(N_c)$, that is,
\begin{equation}
  \label{eq:defkralpha}
  k_{r\alpha}(s, N_c) = m(N_c)\,k_{r\alpha}^{\surf}(s_\alpha, N_c) + (1 - m(N_c))\,k_{r\alpha}^{\nosurf}(s_\alpha, N_c).
\end{equation}
For the adsorption term, we use an approach based on Langmuir isotherm so that the adsorbed concentration is a function of the surfactant/polymer concentration, that is of the from $\cads(c)$. We consider the reversible case and the case without desorption where the term $\cads$ is replaced by
$\hat\cads$ which denotes the maximum value that the adsorption concentration has reached. More precisely, if we define
\begin{equation}
  \label{eq:chatadsmax}
  \chatads_{\max}(t,x) = \max_{t'<t}(\chatads(t',x)),
\end{equation}
then we have defined as
\begin{equation}
  \label{eq:defchatads}
  \chatads(t, x) = \max(\cads(c(t,x)), \chatads_{\max}(t,x)).
\end{equation}

\printbibliography

\end{document}
